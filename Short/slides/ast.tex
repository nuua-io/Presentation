\begin{slide}
    \begin{figure}[H]
        \centering
        \begin{subtable}{0.45\textwidth}
            \begin{minted}{cpp}
1 - 2 * -3
            \end{minted}
            \caption{Program}
        \end{subtable}
        \begin{subfigure}{0.45\textwidth}
            \centering
            \begin{tikzpicture}[node distance=1.75cm]
                \node[state] (1) {\texttt{B(-)}};
                \node[state, below left of=1] (2) {\texttt{1}};
                \node[state, below right of=1] (3) {\texttt{B(*)}};
                \node[state, below left of=3] (4) {\texttt{2}};
                \node[state, below right of=3] (5) {\texttt{U(-)}};
                \node[state, below of=5] (6) {\texttt{3}};
                \draw (1) -- (2);
                \draw (1) -- (3);
                \draw (3) -- (4);
                \draw (3) -- (5);
                \draw (5) -- (6);
            \end{tikzpicture}
            \caption{Nuua AST}
        \end{subfigure}
        \caption{Example abstract syntax tree}
    \end{figure}
\end{slide}

\begin{slide}
    \begin{figure}[H]
        \centering
        \begin{subtable}{0.5\textwidth}
            \begin{minted}{cpp}
fun main(argv: [string]) {
    print 1 - 2 * -3
}
            \end{minted}
            \caption{Program}
        \end{subtable}
        \begin{subfigure}{0.45\textwidth}
            \centering
            \begin{minted}{cpp}
Function[main: <no-return>]
  Print
    Binary[-]
      Integer
      Binary[*]
        Integer
        Unary[-]
          Integer
            \end{minted}
            \caption{Nuua AST}
        \end{subfigure}
        \caption{Code example abstract syntax tree}
    \end{figure}
\end{slide}
