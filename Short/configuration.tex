\documentclass{beamer}
\usetheme[progressbar=frametitle]{metropolis}
% \setbeamertemplate{frame numbering}[fraction]
\useoutertheme{metropolis}
\useinnertheme{metropolis}
\usefonttheme{metropolis}
\definecolor{tw-yellow}{RGB}{236, 201, 75}
\definecolor{tw-indigo}{RGB}{102, 126, 234}
\setbeamercolor{palette primary}{bg=tw-indigo,fg=white}
\setbeamercolor{background canvas}{bg=white}
\setbeamercolor{progress bar}{fg=tw-yellow,bg=white}
% \setbeamercolor{structure}{fg=tw-indigo}
% \setbeamercolor{section in toc}{fg=tw-indigo}
\setbeamercovered{transparent=15}
\setbeamertemplate{section in toc}[sections numbered]
\makeatletter
\setlength{\metropolis@titleseparator@linewidth}{1pt}
\setlength{\metropolis@progressonsectionpage@linewidth}{1pt}
\setlength{\metropolis@progressinheadfoot@linewidth}{1pt}
\makeatother
\metroset{block=fill}
\usepackage[utf8]{inputenc}


% Code listings
\usepackage{listings}
\lstset{
  breaklines=true,
  keepspaces=true,
  basicstyle=\ttfamily
}
\lstset{columns=fullflexible,basicstyle=\ttfamily}
\usepackage{minted}
%\usemintedstyle{colorful}
\setminted{breaklines=true, baselinestretch=1}
%\DeclareBoolOption{newfloat}

% Graphics
\usepackage[labelsep=endash]{caption}
\usepackage{graphicx}
\graphicspath{{images/}}
\usepackage{tikz}
\usetikzlibrary{
    automata,calc,trees,positioning,arrows,chains,shapes.geometric,
    decorations.pathreplacing,decorations.pathmorphing,shapes,
    matrix,shapes.symbols
}
\tikzset{
    block/.style={rectangle, rounded corners, minimum height=3em, draw=black, very thick,, text centered ,text width=5em},
    square/.style={rectangle, draw=black, thick,, text centered},
    big_block/.style={rectangle, rounded corners, minimum height=3em, draw=black, very thick,, text centered ,text width=10em},
    bigger_block/.style={rectangle, rounded corners, minimum height=3em, draw=black, very thick,, text centered ,text width=15em},
    line/.style={->, thick,shorten >=1.5pt},
    dot_arrow/.style={*->, thick, shorten >=1.5pt},
    decoration={brace},
    tuborg/.style={decorate},
    tubnode/.style={midway, right=2pt},
}
\usepackage{newfloat}
\usepackage{subcaption}
\setlength{\abovecaptionskip}{10pt}
\setlength{\belowcaptionskip}{10pt}
\renewcommand{\thesubtable}{\roman{subtable}}
\renewcommand{\thesubfigure}{\roman{subfigure}}
\newenvironment{code}{\captionsetup{type=listing}}{}

% Slides releated
\newcommand{\slidecaps}[1]{\begin{footnotesize}\textsc{#1}\end{footnotesize}}
\newenvironment{slide}{\begin{frame}[fragile, environment=slide]{\thesection.\thesubsection \hphantom{ } \secname : \subsecname}}{\end{frame}}

% Table of contents
\usepackage{etoolbox}
\makeatletter
\patchcmd{\beamer@sectionintoc}{\vskip1.5em}{\vskip0.9em}{}{}
\makeatother

%Information to be included in the title page:
\title[The Nuua Programming Language]{The Design of an Experimental Programming Language and its Translator}
\subtitle{The Nuua Programming Language}
\author{Èrik Campobadal Forés}
\institute{Universitat Politècnica de Catalunya}
\date{\today}
